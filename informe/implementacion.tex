Llegados a este punto, ya familiarizados con la estructura de clases de la aplicación, se procede a la implementación de la misma.

Como dijo el matemático Clive Humby, \enquote{Los datos son el nuevo petróleo}\autocite{talagala2022data}.
En este sentido, la aplicación se ha diseñado para que los datos sean el centro de la misma.
Por ello, se ha implementado un patrón DAO (\textit{Data Access Object}) para abstraer la lógica de acceso a los datos.
Como ya hemos hablado de este patrón en la \autoref{subsec:diagrama_de_clases_dao}, no se repetirá la explicación aquí.
Sin embargo, es un trozo de código interesante para abrir este capítulo.

\section{Código de la aplicación}\label{sec:codigo_de_la_aplicacion}
\subsection{Implementación del DAO}\label{subsec:implementacion_del_dao}

Nada más empezar a analizar este paquete, resalta la existencia de una interfaz \textit{IDAO} que define los métodos básicos de acceso a los datos.
De entrar en ella, se verán dos métodos sobrecargados: \textit{load} y \textit{save}.
Estos métodos son los que se encargan de cargar y guardar los datos, respectivamente.

Algo interesante de esta «sobrecarga» es que solo no se han implementado los métodos sin argumentos.
Es interesante, porque nos permite establecer métodos obligatorios y otros métodos que se han
de implementar si se desea.
Por ejemplo, si se desea implementar un método \textit{load(String file)} que cargue los datos de un archivo específico, se puede hacer.
Pero si no se desea, no es necesario.
He de puntualizar que, esta forma de usar las interfaces fue polémica cuando se implementó\autocite{stackoverflow2014defaultforbidden},
    pero es una forma de hacerlo que me gusta y ha llegado a ser aceptada en la comunidad\autocite{mikemybytes2021defaultmethods}.

Dicho esto, se puede ver que la clase \textit{AbstractDAO} implementa la interfaz \textit{IDAO}.

\subsubsection{Clase AbstractDAO}\label{subsubsec:clase_abstractdao}
Esta clase es el «esqueleto» de los DAOs.
Llegando como llegamos de la interfaz, lo propio sería empezar por los métodos \textit{load} y \textit{save}.
Estos métodos son los que se encargan de cargar y guardar los datos, respectivamente.

Para implementarlos, se ha hecho uso del patrón \textit{Template Method}.
Este patrón es muy útil para definir un esquema de algoritmo y dejar que las subclases implementen los detalles.
Para ejemplificarlo, se puede ver el método \textit{load} en la \autoref{lst:load}.
\begin{lstlisting}[language=Java, caption={Método \textit{load} de la clase \textit{AbstractDAO}}, label={lst:load}]
    @Override
    public boolean load() {
        XMLEventReader reader = null;
        obtenerLogger(); //Obtiene el logger
        logger.info("Cargando archivo...");

        try {
            reader = obtenerXmlEventReader();
            cargarArchivo(reader);
        } catch (CargaArchivoFallidaException e) {
            logger.error("Error al cargar el archivo", e);
            return false;
        } finally {
            try {
                cerrarArchivo(reader);
            } catch (CierreArchivoXMLErroneoException e) {
                logger.error("Error al cerrar el archivo", e);
            }
        }

        logger.info("Archivo cargado de forma satisfactoria");
        return true;
    }
\end{lstlisting}

Este método tiene una lógica muy normal hasta que se llega a la llamada a \textit{obtenerXmlEventReader()}.
Si se mira el código de esta función, se verá que es abstracta.
Uno podría preguntarse, ¿cómo se implementa?
Pues bien, se implementa en las subclases.
De esta forma, se puede tener un método que se ejecuta siempre, pero que depende de la implementación de las subclases.

Luego, se llama a \textit{cargarArchivo(reader)}.
Como la estructura de cada archivo de datos es distinta, este método también es abstracto.
Así permitimos que cada subclase implemente cómo cargar sus datos mientras mantenemos una estructura
común, reducimos la duplicación de código y mantenemos la cohesión.
Este patrón se llama \textit{Template Method} y es muy útil en este tipo de situaciones.\autocite{digitalocean2024templatemethod}.\par

Si seguimos analizando la clase, (obviando el método save, que es similar y los métodos triviales como \textit{obtenerLogger})
llegamos al final, que es una parte interesante.

Estos métodos como \textit{getNextXMLEvent} o \textit{escribirElemento}
son resultado de un proceso de refactorización.
En un principio, estos métodos estaban en las clases hijas, pero
se detectó en el proceso de codificación que estos métodos eran comunes a todas las clases hijas.
Por ello, se decidió moverlos a la clase madre.\par

Analizar una a una las clases hijas sería tedioso y no aportaría mucho.
Por ello, vamos a hacer un análisis general de las clases hijas.
Estas procesan los datos XML\@.
Se ha usado la librería \textit{javax.xml} para procesar los datos XML\@.

Todas las clases hijas implementan el patrón singleton y por ello tienen
un método \textit{getInstance} que devuelve la instancia de la clase.
Se puede ver un ejemplo en \textit{DAOCirculacion}:
\begin{lstlisting}[language=Java, caption={Método \textit{getInstance}}, label={lst:getInstance}]
    public static volatile DAOCirculacion getInstance(FachadaDAO fadao)// Singleton
    {
        if (instance != null)
        {
            return instance;
        }
        synchronized (DAOCirculacion.class)
        {
            if (instance == null)
            {
                instance = new DAOCirculacion(fadao);
            }
        }
    return instance;
    }
\end{lstlisting}

Respecto al resto del código, es bastante trivial.
En el caso de \textit{guardarArchivo}, se ha usado un \textit{BufferedWriter} para escribir los datos.
Con él se itera sobre los datos y se escriben en el archivo.

El caso de \textit{cargarArchivo} es más interesante.
Se ha usado un \textit{XMLEventReader} para leer los datos.
Este itera por los datos con un (o varios) «whiles» y mediante un \textit{enhanced-switch} se procesan los datos.

En resumen, el DAO es una parte fundamental de la aplicación.
Se encarga de la persistencia de los datos y de abstraer la lógica de acceso a los mismos.
En este caso, se ha implementado un patrón DAO con una estructura común y clases hijas que implementan la lógica específica de cada clase.

Las clases de paquetes externos al DAO mismo, acceden a los datos mediante el patrón Façade.

\subsection{Implementación del Façade}\label{subsec:implementacion_del_facade}
Las «façadas» son el «hilo conductor» de la aplicación.
Todo pasa por ellas.
Gracias a ellas, se puede añadir una capa de abstracción entre la GUI y el modelo de datos.
Además, aumenta la encapsulación proporcionada por el MVC\@.
Las façadas implementan también el patrón singleton.

\subsection{Contenedores de datos}\label{subsec:contenedores_de_datos}

Una vez visto el DAO y las façadas, es pertinente hablar de los contenedores de datos.
Estas son las representaciones de como se estructura la información en la aplicación.

Podemos distinguir dos tipos de «contenedores»: los inmutables y los mutables.
En nuestro caso, JAVA nos proporciona un forma de diferenciarlos: los \textit{Records} y las \textit{Clases}.

\subsubsection{Records}\label{subsubsec:records}
En nuestra aplicación, como hay muchos datos de infraestructura y fechas,
hemos acabado teniendo muchos tipos que no admiten cambios.
De esta forma, se ha decidido usar \textit{Records} para representar estos datos.
Estas clases además de ser una «señal» de inmutabilidad,
nos dan alguna que otra ventaja como, por ejemplo, que los «getters», «equals», «hashCode» y «toString» se generan automáticamente.


\subsubsection{Clases}\label{subsubsec:clases}
Al otro lado del espectro, tenemos los usuarios.
Estos pueden cambiar su correo, teléfono y dirección.


\subsection{Implementación de la Interfaz Gráfica de Usuario (GUI)}\label{subsec:implementacion_gui}

La interfaz gráfica de usuario (GUI) es la parte de la aplicación que interactúa con el usuario.
Esta construída con Java Swing.
Hay dos asuntos que son interesantes de comentar: la interacción con el modelo de datos y la construcción de
elementos dependientes de modelos.

\subsubsection{Interacción con el modelo de datos}\label{subsubsec:interaccion_con_el_modelo_de_datos}
La interacción con el modelo de datos se hace mediante las façadas.
Estas se encargan de abstraer la lógica de acceso a los datos.

Por ejemplo, para reservar un tren, se llama al método \textit{reservarTren} de la clase \textit{FachadaAplicacion}:
\begin{lstlisting}[language=Java, caption={Interacción con el modelo de datos}, label={lst:interaccion_modelo_datos}]
    FachadaAplicacion fachada = new FachadaAplicacion(); Usuario usuario = new Usuario(...);
    Tren tren = new Tren(...);
    fachada.reservarTren(usuario, tren);
\end{lstlisting}

\subsubsection{Construcción de elementos dependientes de modelos}\label{subsubsec:construccion_de_elementos_dependientes_de_modelos}
En algunos formularios de la gui, se han usado elementos que, como almacenan información de una forma específica,
necesitan de un modelo específico.

Este es el caso de, por ejemplo, las tablas.
Para construir una tabla, se necesita un modelo de tabla.
Los modelos de tabla son clases que extienden de \textit{AbstractTableModel} y que se encargan de almacenar y gestionar los datos de la tabla.

\subsection{Clases de utilidades}\label{subsec:clases_de_utilidades}
Las clases de utilidades son clases que no tienen estado y que contienen métodos estáticos.
En nuestro caso, tenemos 3: \textit{Criptograficos}, \textit{Internacionalizacion} y \textit{Ortograficos}.

En el caso de \textit{Criptograficos}, se encarga de cifrar y descifrar contraseñas.
En el caso de \textit{Internacionalizacion}, se encarga de cargar los ficheros de internacionalización.
En el caso de \textit{Ortograficos}, se encarga de hacer operaciones de comparación de cadenas.



\section{Problemas encontrados durante la implementación}\label{sec:problemas_encontrados_during_la_implementacion}
Como en todo proyecto, se han encontrado problemas durante la implementación.
Empezando por el principio, el primer problema fue la serialización y persistencia de los datos.

\subsection{Serialización y persistencia de los datos}\label{subsec:serializacion_y_persistencia_de_los_datos}
El primer enfoque que se adoptó para atajar este objetivo fue la serialización que java nos proporciona.
Funcionaba bien, pero los archivos generados eran muy grandes y, no permitían una modificación fuera de la aplicación.
Además, no son agnósticos respecto al lenguaje.

Por ello, se decidió cambiar a XML\@.

Lo que no se esperaba era la dificultad que iba a suponer la lectura y escritura de los datos.
Como se espera trabajar con muchos datos, la carga en memoria del archivo XML no era una opción.
Por ello, había que leer el archivo de forma secuencial y en streaming.

Sin embargo, tras uno o dos refactorings, se consiguió implementar la carga y guardado de los datos de forma eficiente y
con un código más limpio que el de la primera implementación.

\section{Resultados y Conclusiones}\label{sec:resultados_y_conclusiones}
En resumen, la implementación de la aplicación ha sido un éxito.
Se ha implementado un patrón DAO para abstraer la lógica de acceso a los datos.
Se ha implementado un patrón Façade para simplificar la interacción entre la GUI y el modelo de datos.
Se ha implementado un patrón Singleton para las clases DAO y Façade.
Se ha implementado un patrón Template Method para la clase AbstractDAO y sus hijas.
Se han implementado contenedores de datos inmutables y mutables.
Se ha implementado la GUI con Java Swing.
Se ha creado una clase para cifrar y descifrar contraseñas.
Se han encontrado problemas durante la implementación, pero se han resuelto con éxito.
Se ha implementado la persistencia de los datos en XML\@.
Se ha permitido la internacionalización de la aplicación.
Se ha implementado un sistema de loggin.

Toda esta funcionalidad no significa tampoco que el proyecto esté perfecto,
quedan algunas cosas que habría sido interesante implementar.

\section{Trabajo futuro}\label{sec:trabajo_futuro}
Como «deberes» hipotéticos para el futuro, se podrían implementar las siguientes funcionalidades:
\begin{itemize}
\item Permitir cambiar la contraseña de un usuario.
    \item Filtrar los trenes por fecha y hora.
    \item Filtrar los trenes por precio
    \item No mostrar los trenes que ya han pasado o que estén cancelados.
    \item Tal vez, migrar a una base de datos.
    \item De migrar a una base de datos, añadir protección contra conflictos de concurrencia.
\end{itemize}







